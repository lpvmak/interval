\include{settings}
\usepackage{amsmath}

\lstset{language=Java} 

\begin{document}	% начало документа

% Титульная страница
\include{titlepage}

% Содержание
\renewcommand\contentsname{\centerline{Содержание}}
\tableofcontents
\newpage


\section{Постановка задачи}

\subsection{Задача 1}

Имеем 2$\times$2-матрицу $A$

\begin{equation}
	A = \begin{pmatrix}
		1 & 1 \\
		1.1 & 1
		\end{pmatrix}
\end{equation}

Пусть все элементы матрицы $a_{ij}$ имеют радиус $\varepsilon$:

\begin{equation}
	rad \textbf{a}_{ij} = \varepsilon
\end{equation}

Получаем 
\begin{equation}
	\textbf{A} = \begin{pmatrix}
		[1 - \varepsilon, 1 + \varepsilon] & [1 - \varepsilon, 1 + \varepsilon] \\
		[1.1 - \varepsilon, 1.1 + \varepsilon] & [1 - \varepsilon, 1 + \varepsilon]
	\end{pmatrix}
\end{equation}

Определить, при каком радиусе $\varepsilon$ матрица $\textbf{A}$ содержит особенные матрицы.

\subsection{Задача 2}

Имеем $n \times n$-матрицу $\textbf{A}$ 

\begin{equation}
	\textbf{A} = \begin{pmatrix}
		1 & [0,\varepsilon] & \dots & [0,\varepsilon] \\
		[0,\varepsilon] & 1 & \dots & [0,\varepsilon] \\
		\dots & \dots & \dots & \dots \\
		[0,\varepsilon] & [0,\varepsilon] & \dots & 1 
	\end{pmatrix}
\end{equation}

Определить, при каком радиусе $\varepsilon$ матрица $\textbf{A}$ содержит особенные матрицы.



\section{Решение}

\subsection{Задача 1}

\subsubsection{Аналитический вывод}

Воспользуемся критерием Баумана \\

\textbf{Теорема} \\
Интервальная матрица $A$ неособенна тогда и только тогда, когда определители всех ее крайних матриц имеют одинаковый знак, т.е.

\begin{equation}
	(\det A') * (\det A'') > 0
\end{equation}

для любых $A', A'' \in vert A$. \\

Найдем, при каких значениях $\varepsilon$ матрица является неособенной. Тогда особенной она будет при всех остальных значениях. \\
Посчитаем определители всех крайних матриц, т.е. получим 16 функций, зависящих от $\varepsilon$. Некоторые функции повторяются, поэтому их опустим:
\begin{subequations} \label{eq}
	\begin{align}
		-0.1 (1 - \varepsilon)	\label{eq:1} \\
		-0.1 (1 + \varepsilon)	\label{eq:2} \\
		4.1 \varepsilon - 0.1	\label{eq:3} \\
		-4.1 \varepsilon - 0.1	\label{eq:4} \\
		(1 - \varepsilon) (2 \varepsilon - 0.1)	\label{eq:5} \\
		(1 - \varepsilon) (-2 \varepsilon - 0.1)	\label{eq:6} \\
		(1 + \varepsilon) (2 \varepsilon - 0.1)	\label{eq:7} \\
		(1 + \varepsilon) (-2 \varepsilon - 0.1)	\label{eq:8} \\
		2\varepsilon^{2} - 2.1\varepsilon - 0.1	\label{eq:9} \\
		2\varepsilon^{2} + 2.1\varepsilon - 0.1	\label{eq:10}
	\end{align}
\end{subequations}

Так как $\varepsilon \geq 0$, то из (\ref{eq:2}) получаем, что определители всех крайних матриц должны быть отрицательными. \\
Запишем выведенные из выражений (\ref{eq}) ограничения на $\varepsilon$. Неравенства, где $\varepsilon$ должен быть больше некоторого отрицательного числа, опустим.

\begin{subequations}
	\begin{align}
		\varepsilon < 1	\label{eps:1} \\
		\varepsilon < \frac{1}{20}	\label{eps:2} \\
		\varepsilon < 1.09	\label{eps:3} \\
		\varepsilon < \frac{1}{40}	\label{eps:4} \\
		\varepsilon < 0.0456	\label{eps:5}
	\end{align}
\end{subequations}

Минимальная правая граница равна $\frac{1}{41}$. Максимальная левая граница - 0. Значит, матрица $\textbf{A}$ является неособенной при $\varepsilon \in [0, \frac{1}{41})$. \\

Соответственно, особенной матрица является при $\varepsilon \geq \frac{1}{41}$. \\

Примером особенной матрицы, у которой значение $\varepsilon$ близко к границе, может служить данная матрица ($\varepsilon = \frac{1}{40}$):

\begin{equation}
	\textbf{A} = \begin{pmatrix}
		[0.975, 1.025] & [0.975, 1.0.25] \\
		[1.075, 1.125] & [0.975, 1.0.25]
	\end{pmatrix}
\end{equation}

Определитель этой матрицы равен: $[-0.2025, 0.0025]$

\subsubsection{Численный эксперимент}

Проверим, что полученные значения $\varepsilon$ соответствуют особенным и неособенным матрицам. Для этого посчитаем определитель-интервал при следующих значениях радиуса:
\begin{equation}
	[0, \frac{1}{70}, \frac{1}{42}, \frac{1}{41}, \frac{1}{40}, 1]
\end{equation}
Получим следующие результаты:

\newpage

\begin{figure}[h]
	\centering
	\includegraphics[width=0.75\textwidth]{task_1.png}
	\caption{Определители матриц с заданными $\varepsilon$}
\end{figure}

Результаты численного эксперимента соответствуют полученным аналитически результатам: для значений $[0, \frac{1}{70}, \frac{1}{42}]$ определитель не содержит 0; для значений $[\frac{1}{41}, \frac{1}{40}, 1]$ определитель содержит 0.


\subsection{Задача 2}

\subsubsection{Численный эксперимент}

Очевидно, что при $\varepsilon \geq 1$ матрица является особенной, так как содержит особенную точечную матрицу со всеми элементами - единицами. \\
Попробуем улучшить эту оценку, используя признаки Бекка и Румпа. \\

\textbf{Теорема (признак Бекка)} \\
Пусть интервальная матрица $A \in IR^{n \times n}$ такова, что ее середина $mid A$ неособенна и
\begin{equation}
	\rho(|(mid A)^{-1}| * rad A) < 1
\end{equation}
Тогда A неособенна. \\

\textbf{Теорема (признак Румпа)} \\
Если для интервальной матрицы $A \in IR^{n \times n}$ имеет место
\begin{equation}
	\sigma_{max}(rad A) < \sigma_{min}(mid A)
\end{equation}
Тогда A неособенна. \\

Возьмем начальное значение $\varepsilon = 0.001$ и будем его увеличивать с шагом $0.001$, пока хотя бы один из признаков подтверждает, что матрица является неособенной. Запишем $\varepsilon$ для такой последней матрицы.\\
Размерности матриц возьмем от 2 до 20. \\
Получим следующие результаты:

\begin{figure}[h]
	\centering
	\includegraphics[width=0.75\textwidth]{task_2_1.png}
	\caption{Полученные значения $\varepsilon$ и величины $\frac{1}{n - 1}$}
\end{figure}

\newpage

\begin{figure}[h]
	\centering
	\includegraphics[width=0.75\textwidth]{task_2_2.png}
	\caption{Полученные значения $\varepsilon$ и величины $\frac{1}{n - 1}$}
\end{figure}

\newpage

\begin{figure}[h]
	\centering
	\includegraphics[width=0.75\textwidth]{task_2_3.png}
	\caption{Полученные значения $\varepsilon$ и величины $\frac{1}{n - 1}$}
\end{figure}

\newpage

\begin{figure}[h]
	\centering
	\includegraphics[width=0.75\textwidth]{task_2_4.png}
	\caption{Полученные значения $\varepsilon$ и величины $\frac{1}{n - 1}$}
\end{figure}

Запишем результаты в виде таблицы:

\begin{center}
	\begin{tabular}{ |c|c|c| } 
		\hline
		n & $\frac{1}{n - 1}$ & $\varepsilon$ \\ 
		\hline
		2 & 1.0 & 0.666 \\ 
		3 & 0.5 & 0.399 \\ 
		4 & 0.333 & 0.285 \\ 
		5 & 0.25 & 0.222 \\ 
		6 & 0.2 & 0.181 \\ 
		7 & 0.166 & 0.153 \\ 
		8 & 0.142 & 0.133 \\ 
		9 & 0.125 & 0.117 \\ 
		10 & 0.111 & 0.111 \\ 
		11 & 0.1 & 0.095 \\ 
		12 & 0.09 & 0.086 \\ 
		13 & 0.083 & 0.079 \\ 
		14 & 0.076 & 0.074 \\ 
		15 & 0.071 & 0.068 \\ 
		16 & 0.066 & 0.064 \\ 
		17 & 0.0625 & 0.06 \\ 
		18 & 0.058 & 0.057 \\ 
		19 & 0.055 & 0.054 \\ 
		20 & 0.052 & 0.051 \\ 
		\hline
	\end{tabular}
\end{center}

Из таблицы видно, что полученные $\varepsilon$ немного меньше величин $\frac{1}{n - 1}$, которые можно получить, применив признак Адамара. \\

\textbf{Теорема (интервальный признак Адамара)} \\
Интервальная матрица с диагональным преобладанием является неособенной. \\

Таким образом, можно использовать значения $\varepsilon > \frac{1}{n - 1}$ для получения особенных матриц.


\section{Приложение}
Код программы на Python лежит в данном репозитории: \\
\url{https://github.com/PinkOink/Interval_Analysis/tree/main/lab1}{}


\end{document}